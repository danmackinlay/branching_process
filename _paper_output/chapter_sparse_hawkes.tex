
% Default to the notebook output style

    


% Inherit from the specified cell style.




    

\documentclass[11pt]{article}


    
    

    \usepackage[T1]{fontenc}
    % Nicer default font (+ math font) than Computer Modern for most use cases
    \usepackage{mathpazo}

    % Basic figure setup, for now with no caption control since it's done
    % automatically by Pandoc (which extracts ![](path) syntax from Markdown).
    \usepackage{graphicx}
    % We will generate all images so they have a width \maxwidth. This means
    % that they will get their normal width if they fit onto the page, but
    % are scaled down if they would overflow the margins.
    \makeatletter
    \def\maxwidth{\ifdim\Gin@nat@width>\linewidth\linewidth
    \else\Gin@nat@width\fi}
    \makeatother
    \let\Oldincludegraphics\includegraphics
    % Set max figure width to be 80% of text width, for now hardcoded.
    \renewcommand{\includegraphics}[1]{\Oldincludegraphics[width=.8\maxwidth]{#1}}
    % Ensure that by default, figures have no caption (until we provide a
    % proper Figure object with a Caption API and a way to capture that
    % in the conversion process - todo).
    \usepackage{caption}
    \DeclareCaptionLabelFormat{nolabel}{}
    \captionsetup{labelformat=nolabel}

    \usepackage{adjustbox} % Used to constrain images to a maximum size 
    \usepackage{xcolor} % Allow colors to be defined
    \usepackage{enumerate} % Needed for markdown enumerations to work
    \usepackage{geometry} % Used to adjust the document margins
    \usepackage{amsmath} % Equations
    \usepackage{amssymb} % Equations
    \usepackage{textcomp} % defines textquotesingle
    % Hack from http://tex.stackexchange.com/a/47451/13684:
    \AtBeginDocument{%
        \def\PYZsq{\textquotesingle}% Upright quotes in Pygmentized code
    }
    \usepackage{upquote} % Upright quotes for verbatim code
    \usepackage{eurosym} % defines \euro
    \usepackage[mathletters]{ucs} % Extended unicode (utf-8) support
    \usepackage[utf8x]{inputenc} % Allow utf-8 characters in the tex document
    \usepackage{fancyvrb} % verbatim replacement that allows latex
    \usepackage{grffile} % extends the file name processing of package graphics 
                         % to support a larger range 
    % The hyperref package gives us a pdf with properly built
    % internal navigation ('pdf bookmarks' for the table of contents,
    % internal cross-reference links, web links for URLs, etc.)
    \usepackage{hyperref}
    \usepackage{longtable} % longtable support required by pandoc >1.10
    \usepackage{booktabs}  % table support for pandoc > 1.12.2
    \usepackage[inline]{enumitem} % IRkernel/repr support (it uses the enumerate* environment)
    \usepackage[normalem]{ulem} % ulem is needed to support strikethroughs (\sout)
                                % normalem makes italics be italics, not underlines
    
\usepackage{apacite}


    
    
    % Colors for the hyperref package
    \definecolor{urlcolor}{rgb}{0,.145,.698}
    \definecolor{linkcolor}{rgb}{.71,0.21,0.01}
    \definecolor{citecolor}{rgb}{.12,.54,.11}

    % ANSI colors
    \definecolor{ansi-black}{HTML}{3E424D}
    \definecolor{ansi-black-intense}{HTML}{282C36}
    \definecolor{ansi-red}{HTML}{E75C58}
    \definecolor{ansi-red-intense}{HTML}{B22B31}
    \definecolor{ansi-green}{HTML}{00A250}
    \definecolor{ansi-green-intense}{HTML}{007427}
    \definecolor{ansi-yellow}{HTML}{DDB62B}
    \definecolor{ansi-yellow-intense}{HTML}{B27D12}
    \definecolor{ansi-blue}{HTML}{208FFB}
    \definecolor{ansi-blue-intense}{HTML}{0065CA}
    \definecolor{ansi-magenta}{HTML}{D160C4}
    \definecolor{ansi-magenta-intense}{HTML}{A03196}
    \definecolor{ansi-cyan}{HTML}{60C6C8}
    \definecolor{ansi-cyan-intense}{HTML}{258F8F}
    \definecolor{ansi-white}{HTML}{C5C1B4}
    \definecolor{ansi-white-intense}{HTML}{A1A6B2}

    % commands and environments needed by pandoc snippets
    % extracted from the output of `pandoc -s`
    \providecommand{\tightlist}{%
      \setlength{\itemsep}{0pt}\setlength{\parskip}{0pt}}
    \DefineVerbatimEnvironment{Highlighting}{Verbatim}{commandchars=\\\{\}}
    % Add ',fontsize=\small' for more characters per line
    \newenvironment{Shaded}{}{}
    \newcommand{\KeywordTok}[1]{\textcolor[rgb]{0.00,0.44,0.13}{\textbf{{#1}}}}
    \newcommand{\DataTypeTok}[1]{\textcolor[rgb]{0.56,0.13,0.00}{{#1}}}
    \newcommand{\DecValTok}[1]{\textcolor[rgb]{0.25,0.63,0.44}{{#1}}}
    \newcommand{\BaseNTok}[1]{\textcolor[rgb]{0.25,0.63,0.44}{{#1}}}
    \newcommand{\FloatTok}[1]{\textcolor[rgb]{0.25,0.63,0.44}{{#1}}}
    \newcommand{\CharTok}[1]{\textcolor[rgb]{0.25,0.44,0.63}{{#1}}}
    \newcommand{\StringTok}[1]{\textcolor[rgb]{0.25,0.44,0.63}{{#1}}}
    \newcommand{\CommentTok}[1]{\textcolor[rgb]{0.38,0.63,0.69}{\textit{{#1}}}}
    \newcommand{\OtherTok}[1]{\textcolor[rgb]{0.00,0.44,0.13}{{#1}}}
    \newcommand{\AlertTok}[1]{\textcolor[rgb]{1.00,0.00,0.00}{\textbf{{#1}}}}
    \newcommand{\FunctionTok}[1]{\textcolor[rgb]{0.02,0.16,0.49}{{#1}}}
    \newcommand{\RegionMarkerTok}[1]{{#1}}
    \newcommand{\ErrorTok}[1]{\textcolor[rgb]{1.00,0.00,0.00}{\textbf{{#1}}}}
    \newcommand{\NormalTok}[1]{{#1}}
    
    % Additional commands for more recent versions of Pandoc
    \newcommand{\ConstantTok}[1]{\textcolor[rgb]{0.53,0.00,0.00}{{#1}}}
    \newcommand{\SpecialCharTok}[1]{\textcolor[rgb]{0.25,0.44,0.63}{{#1}}}
    \newcommand{\VerbatimStringTok}[1]{\textcolor[rgb]{0.25,0.44,0.63}{{#1}}}
    \newcommand{\SpecialStringTok}[1]{\textcolor[rgb]{0.73,0.40,0.53}{{#1}}}
    \newcommand{\ImportTok}[1]{{#1}}
    \newcommand{\DocumentationTok}[1]{\textcolor[rgb]{0.73,0.13,0.13}{\textit{{#1}}}}
    \newcommand{\AnnotationTok}[1]{\textcolor[rgb]{0.38,0.63,0.69}{\textbf{\textit{{#1}}}}}
    \newcommand{\CommentVarTok}[1]{\textcolor[rgb]{0.38,0.63,0.69}{\textbf{\textit{{#1}}}}}
    \newcommand{\VariableTok}[1]{\textcolor[rgb]{0.10,0.09,0.49}{{#1}}}
    \newcommand{\ControlFlowTok}[1]{\textcolor[rgb]{0.00,0.44,0.13}{\textbf{{#1}}}}
    \newcommand{\OperatorTok}[1]{\textcolor[rgb]{0.40,0.40,0.40}{{#1}}}
    \newcommand{\BuiltInTok}[1]{{#1}}
    \newcommand{\ExtensionTok}[1]{{#1}}
    \newcommand{\PreprocessorTok}[1]{\textcolor[rgb]{0.74,0.48,0.00}{{#1}}}
    \newcommand{\AttributeTok}[1]{\textcolor[rgb]{0.49,0.56,0.16}{{#1}}}
    \newcommand{\InformationTok}[1]{\textcolor[rgb]{0.38,0.63,0.69}{\textbf{\textit{{#1}}}}}
    \newcommand{\WarningTok}[1]{\textcolor[rgb]{0.38,0.63,0.69}{\textbf{\textit{{#1}}}}}
    
    
    % Define a nice break command that doesn't care if a line doesn't already
    % exist.
    \def\br{\hspace*{\fill} \\* }
    % Math Jax compatability definitions
    \def\gt{>}
    \def\lt{<}
    % Document parameters
    
\title{Regularised parameter selection in the non-stationary Hawkes process}

    



    

\author{Daniel MacKinlay}

    

    % Pygments definitions
    
\makeatletter
\def\PY@reset{\let\PY@it=\relax \let\PY@bf=\relax%
    \let\PY@ul=\relax \let\PY@tc=\relax%
    \let\PY@bc=\relax \let\PY@ff=\relax}
\def\PY@tok#1{\csname PY@tok@#1\endcsname}
\def\PY@toks#1+{\ifx\relax#1\empty\else%
    \PY@tok{#1}\expandafter\PY@toks\fi}
\def\PY@do#1{\PY@bc{\PY@tc{\PY@ul{%
    \PY@it{\PY@bf{\PY@ff{#1}}}}}}}
\def\PY#1#2{\PY@reset\PY@toks#1+\relax+\PY@do{#2}}

\expandafter\def\csname PY@tok@w\endcsname{\def\PY@tc##1{\textcolor[rgb]{0.73,0.73,0.73}{##1}}}
\expandafter\def\csname PY@tok@c\endcsname{\let\PY@it=\textit\def\PY@tc##1{\textcolor[rgb]{0.25,0.50,0.50}{##1}}}
\expandafter\def\csname PY@tok@cp\endcsname{\def\PY@tc##1{\textcolor[rgb]{0.74,0.48,0.00}{##1}}}
\expandafter\def\csname PY@tok@k\endcsname{\let\PY@bf=\textbf\def\PY@tc##1{\textcolor[rgb]{0.00,0.50,0.00}{##1}}}
\expandafter\def\csname PY@tok@kp\endcsname{\def\PY@tc##1{\textcolor[rgb]{0.00,0.50,0.00}{##1}}}
\expandafter\def\csname PY@tok@kt\endcsname{\def\PY@tc##1{\textcolor[rgb]{0.69,0.00,0.25}{##1}}}
\expandafter\def\csname PY@tok@o\endcsname{\def\PY@tc##1{\textcolor[rgb]{0.40,0.40,0.40}{##1}}}
\expandafter\def\csname PY@tok@ow\endcsname{\let\PY@bf=\textbf\def\PY@tc##1{\textcolor[rgb]{0.67,0.13,1.00}{##1}}}
\expandafter\def\csname PY@tok@nb\endcsname{\def\PY@tc##1{\textcolor[rgb]{0.00,0.50,0.00}{##1}}}
\expandafter\def\csname PY@tok@nf\endcsname{\def\PY@tc##1{\textcolor[rgb]{0.00,0.00,1.00}{##1}}}
\expandafter\def\csname PY@tok@nc\endcsname{\let\PY@bf=\textbf\def\PY@tc##1{\textcolor[rgb]{0.00,0.00,1.00}{##1}}}
\expandafter\def\csname PY@tok@nn\endcsname{\let\PY@bf=\textbf\def\PY@tc##1{\textcolor[rgb]{0.00,0.00,1.00}{##1}}}
\expandafter\def\csname PY@tok@ne\endcsname{\let\PY@bf=\textbf\def\PY@tc##1{\textcolor[rgb]{0.82,0.25,0.23}{##1}}}
\expandafter\def\csname PY@tok@nv\endcsname{\def\PY@tc##1{\textcolor[rgb]{0.10,0.09,0.49}{##1}}}
\expandafter\def\csname PY@tok@no\endcsname{\def\PY@tc##1{\textcolor[rgb]{0.53,0.00,0.00}{##1}}}
\expandafter\def\csname PY@tok@nl\endcsname{\def\PY@tc##1{\textcolor[rgb]{0.63,0.63,0.00}{##1}}}
\expandafter\def\csname PY@tok@ni\endcsname{\let\PY@bf=\textbf\def\PY@tc##1{\textcolor[rgb]{0.60,0.60,0.60}{##1}}}
\expandafter\def\csname PY@tok@na\endcsname{\def\PY@tc##1{\textcolor[rgb]{0.49,0.56,0.16}{##1}}}
\expandafter\def\csname PY@tok@nt\endcsname{\let\PY@bf=\textbf\def\PY@tc##1{\textcolor[rgb]{0.00,0.50,0.00}{##1}}}
\expandafter\def\csname PY@tok@nd\endcsname{\def\PY@tc##1{\textcolor[rgb]{0.67,0.13,1.00}{##1}}}
\expandafter\def\csname PY@tok@s\endcsname{\def\PY@tc##1{\textcolor[rgb]{0.73,0.13,0.13}{##1}}}
\expandafter\def\csname PY@tok@sd\endcsname{\let\PY@it=\textit\def\PY@tc##1{\textcolor[rgb]{0.73,0.13,0.13}{##1}}}
\expandafter\def\csname PY@tok@si\endcsname{\let\PY@bf=\textbf\def\PY@tc##1{\textcolor[rgb]{0.73,0.40,0.53}{##1}}}
\expandafter\def\csname PY@tok@se\endcsname{\let\PY@bf=\textbf\def\PY@tc##1{\textcolor[rgb]{0.73,0.40,0.13}{##1}}}
\expandafter\def\csname PY@tok@sr\endcsname{\def\PY@tc##1{\textcolor[rgb]{0.73,0.40,0.53}{##1}}}
\expandafter\def\csname PY@tok@ss\endcsname{\def\PY@tc##1{\textcolor[rgb]{0.10,0.09,0.49}{##1}}}
\expandafter\def\csname PY@tok@sx\endcsname{\def\PY@tc##1{\textcolor[rgb]{0.00,0.50,0.00}{##1}}}
\expandafter\def\csname PY@tok@m\endcsname{\def\PY@tc##1{\textcolor[rgb]{0.40,0.40,0.40}{##1}}}
\expandafter\def\csname PY@tok@gh\endcsname{\let\PY@bf=\textbf\def\PY@tc##1{\textcolor[rgb]{0.00,0.00,0.50}{##1}}}
\expandafter\def\csname PY@tok@gu\endcsname{\let\PY@bf=\textbf\def\PY@tc##1{\textcolor[rgb]{0.50,0.00,0.50}{##1}}}
\expandafter\def\csname PY@tok@gd\endcsname{\def\PY@tc##1{\textcolor[rgb]{0.63,0.00,0.00}{##1}}}
\expandafter\def\csname PY@tok@gi\endcsname{\def\PY@tc##1{\textcolor[rgb]{0.00,0.63,0.00}{##1}}}
\expandafter\def\csname PY@tok@gr\endcsname{\def\PY@tc##1{\textcolor[rgb]{1.00,0.00,0.00}{##1}}}
\expandafter\def\csname PY@tok@ge\endcsname{\let\PY@it=\textit}
\expandafter\def\csname PY@tok@gs\endcsname{\let\PY@bf=\textbf}
\expandafter\def\csname PY@tok@gp\endcsname{\let\PY@bf=\textbf\def\PY@tc##1{\textcolor[rgb]{0.00,0.00,0.50}{##1}}}
\expandafter\def\csname PY@tok@go\endcsname{\def\PY@tc##1{\textcolor[rgb]{0.53,0.53,0.53}{##1}}}
\expandafter\def\csname PY@tok@gt\endcsname{\def\PY@tc##1{\textcolor[rgb]{0.00,0.27,0.87}{##1}}}
\expandafter\def\csname PY@tok@err\endcsname{\def\PY@bc##1{\setlength{\fboxsep}{0pt}\fcolorbox[rgb]{1.00,0.00,0.00}{1,1,1}{\strut ##1}}}
\expandafter\def\csname PY@tok@kc\endcsname{\let\PY@bf=\textbf\def\PY@tc##1{\textcolor[rgb]{0.00,0.50,0.00}{##1}}}
\expandafter\def\csname PY@tok@kd\endcsname{\let\PY@bf=\textbf\def\PY@tc##1{\textcolor[rgb]{0.00,0.50,0.00}{##1}}}
\expandafter\def\csname PY@tok@kn\endcsname{\let\PY@bf=\textbf\def\PY@tc##1{\textcolor[rgb]{0.00,0.50,0.00}{##1}}}
\expandafter\def\csname PY@tok@kr\endcsname{\let\PY@bf=\textbf\def\PY@tc##1{\textcolor[rgb]{0.00,0.50,0.00}{##1}}}
\expandafter\def\csname PY@tok@bp\endcsname{\def\PY@tc##1{\textcolor[rgb]{0.00,0.50,0.00}{##1}}}
\expandafter\def\csname PY@tok@fm\endcsname{\def\PY@tc##1{\textcolor[rgb]{0.00,0.00,1.00}{##1}}}
\expandafter\def\csname PY@tok@vc\endcsname{\def\PY@tc##1{\textcolor[rgb]{0.10,0.09,0.49}{##1}}}
\expandafter\def\csname PY@tok@vg\endcsname{\def\PY@tc##1{\textcolor[rgb]{0.10,0.09,0.49}{##1}}}
\expandafter\def\csname PY@tok@vi\endcsname{\def\PY@tc##1{\textcolor[rgb]{0.10,0.09,0.49}{##1}}}
\expandafter\def\csname PY@tok@vm\endcsname{\def\PY@tc##1{\textcolor[rgb]{0.10,0.09,0.49}{##1}}}
\expandafter\def\csname PY@tok@sa\endcsname{\def\PY@tc##1{\textcolor[rgb]{0.73,0.13,0.13}{##1}}}
\expandafter\def\csname PY@tok@sb\endcsname{\def\PY@tc##1{\textcolor[rgb]{0.73,0.13,0.13}{##1}}}
\expandafter\def\csname PY@tok@sc\endcsname{\def\PY@tc##1{\textcolor[rgb]{0.73,0.13,0.13}{##1}}}
\expandafter\def\csname PY@tok@dl\endcsname{\def\PY@tc##1{\textcolor[rgb]{0.73,0.13,0.13}{##1}}}
\expandafter\def\csname PY@tok@s2\endcsname{\def\PY@tc##1{\textcolor[rgb]{0.73,0.13,0.13}{##1}}}
\expandafter\def\csname PY@tok@sh\endcsname{\def\PY@tc##1{\textcolor[rgb]{0.73,0.13,0.13}{##1}}}
\expandafter\def\csname PY@tok@s1\endcsname{\def\PY@tc##1{\textcolor[rgb]{0.73,0.13,0.13}{##1}}}
\expandafter\def\csname PY@tok@mb\endcsname{\def\PY@tc##1{\textcolor[rgb]{0.40,0.40,0.40}{##1}}}
\expandafter\def\csname PY@tok@mf\endcsname{\def\PY@tc##1{\textcolor[rgb]{0.40,0.40,0.40}{##1}}}
\expandafter\def\csname PY@tok@mh\endcsname{\def\PY@tc##1{\textcolor[rgb]{0.40,0.40,0.40}{##1}}}
\expandafter\def\csname PY@tok@mi\endcsname{\def\PY@tc##1{\textcolor[rgb]{0.40,0.40,0.40}{##1}}}
\expandafter\def\csname PY@tok@il\endcsname{\def\PY@tc##1{\textcolor[rgb]{0.40,0.40,0.40}{##1}}}
\expandafter\def\csname PY@tok@mo\endcsname{\def\PY@tc##1{\textcolor[rgb]{0.40,0.40,0.40}{##1}}}
\expandafter\def\csname PY@tok@ch\endcsname{\let\PY@it=\textit\def\PY@tc##1{\textcolor[rgb]{0.25,0.50,0.50}{##1}}}
\expandafter\def\csname PY@tok@cm\endcsname{\let\PY@it=\textit\def\PY@tc##1{\textcolor[rgb]{0.25,0.50,0.50}{##1}}}
\expandafter\def\csname PY@tok@cpf\endcsname{\let\PY@it=\textit\def\PY@tc##1{\textcolor[rgb]{0.25,0.50,0.50}{##1}}}
\expandafter\def\csname PY@tok@c1\endcsname{\let\PY@it=\textit\def\PY@tc##1{\textcolor[rgb]{0.25,0.50,0.50}{##1}}}
\expandafter\def\csname PY@tok@cs\endcsname{\let\PY@it=\textit\def\PY@tc##1{\textcolor[rgb]{0.25,0.50,0.50}{##1}}}

\def\PYZbs{\char`\\}
\def\PYZus{\char`\_}
\def\PYZob{\char`\{}
\def\PYZcb{\char`\}}
\def\PYZca{\char`\^}
\def\PYZam{\char`\&}
\def\PYZlt{\char`\<}
\def\PYZgt{\char`\>}
\def\PYZsh{\char`\#}
\def\PYZpc{\char`\%}
\def\PYZdl{\char`\$}
\def\PYZhy{\char`\-}
\def\PYZsq{\char`\'}
\def\PYZdq{\char`\"}
\def\PYZti{\char`\~}
% for compatibility with earlier versions
\def\PYZat{@}
\def\PYZlb{[}
\def\PYZrb{]}
\makeatother


    % Exact colors from NB
    \definecolor{incolor}{rgb}{0.0, 0.0, 0.5}
    \definecolor{outcolor}{rgb}{0.545, 0.0, 0.0}



    

    % Prevent overflowing lines due to hard-to-break entities
    \sloppy 
    % Setup hyperref package
    \hypersetup{
      breaklinks=true,  % so long urls are correctly broken across lines
      colorlinks=true,
      urlcolor=urlcolor,
      linkcolor=linkcolor,
      citecolor=citecolor,
      }
    % Slightly bigger margins than the latex defaults
    
    \geometry{verbose,tmargin=1in,bmargin=1in,lmargin=1in,rmargin=1in}
    
    


    \begin{document}
    
    
    \maketitle
    
    

    
    \section{\texorpdfstring{\(\ell_1\)-penalised non-stationary Hawkes
branching
process}{\textbackslash{}ell\_1-penalised non-stationary Hawkes branching process}}\label{ell_1-penalised-non-stationary-hawkes-branching-process}





    \subsection{Abstract}\label{abstract}

We are interested in the problem of simultaneous identification of the

\begin{enumerate}
\def\labelenumi{\arabic{enumi}.}
\tightlist
\item
  self excitation kernel, and
\item
  (possibly inhomogenous) background rate
\end{enumerate}

of a Hawkes point process a using of \(\ell_1\) penalised regression.

This is analaogous to a blind deconvolution problem
\cite{stockham_blind_1975}, adaptive robust regression
\cite{donoho_automatic_1988}, or a sparse inverse problem
\cite{tibshirani_regression_1996}, or, but considerably more
challenging; While the process model superficially resembles a linear
system, there are few of the convenient results that would apply in a
linear regression problem; and as a time series problem it is difficult
to choose our penalty parameters by standard non-parametric means such
as cross validation. The proposed method uses a heuristically-motivated
degrees-of-freedom regularisation parameter selection, which we test
through simulation. The proposed method works well in practice.

    \subsection{Introduction}\label{introduction}

\subsubsection{Notation}\label{notation}

We are given a counting process \(N(t)\) on \(\mathbb R\)

We represent a realisation of the time series by a (finite by
assumption) sequence of arrival times
\(\mathbf{t}=\{\dots,t_1, t_2,\dots,t_N(T)\}.\) We will be concerned
with estimating its likelihood based on the likelihood of the series on
an interval an interval \([0,T]\). We define
\(\mathbf{t}':=\mathbf{t}\cap [0,T]\) and assume
\(\left|\mathbf{t}'\right|<\infty.\)

We also, when notationally convenient, use the equivalent representation
as the left-continuous function that increments at each arrival time: \[
N(t)-N(s):= \left|\{s:s\in \mathbf{t}\,\mathrm{and}\,s\leq t\}\right| \,\mathrm{ for }\, t\geq s
\]

We set \(\lim_{s\rightarrow 0} N(s)\equiv 0\) and
\(t_1\equiv \inf \mathbf{t}'\); Indices \(i\leq 0\) are reserved for
point in the "history" of the process before time \(0\), which may be
generated by any arbitrary process. We will not be concerned with
evaluating the likelihood of such histories but may retain them to
control for their influence on the future of the series. For the rates
to be well-defined, however, we will assume that the increment sof the
process are finite on any finite interval.

We use only the left-continuous filtration
\(\mathcal{F}(\{N(s)\}_{s < t})\) generated by the process, which we
here write as \(\mathcal{F}(N(t)).\)

We call it a regular point process if it has an associated
\emph{intensity} process \(\lambda\), such that \(\lambda\) is
predictable with respect to \(\mathcal{F}(N)\) generated by the process,
i.e. \cite{daley_introduction_2003}

\[\lambda(t|\mathcal{F}(N(t))):=\lim_{h\to 0} \frac{\mathrm E\left(N(t,t+h)\right)}{h}.\]

Hereafter we define \(\lambda^*(t):=\lambda(t|\mathcal{F}(N(t)))\) for
brevity.

Such regular point processes generalise the homogeneous Poisson rate in
the sense that if we choose \(\lambda*(t)\equiv \lambda_0\) then it is
the intensity of a homogeneous Poisson rate with fixed intensity
\(\lambda_0\).

    \[N(t)\sim \operatorname{Pois}(\Lambda(t|\mathcal{F}(N(t)))\] where
\(\Lambda(s,t|\mathcal{F}(N(t)))=\int_s^t \lambda(s|\mathcal{F}(N(s))).\)

The waiting times are exponentially distributed given the intensity,
i.e. their densities \(f_i(s)\) are given \[\begin{aligned}
f_i(t)&\equiv f(t_{i+1}-t_i|\mathcal{F}(N(t_i)))\\
&=\Lambda(t_i,t|\mathcal{F}(N(t_i))\exp\left(-t\Lambda(t_i,t|\mathcal{F}(N(t_i))\right)
\end{aligned}\]

We will use the for the inter-arrival waiting times \(s_i=t_{i+1}-t_i\)
to calculate the data likelihood, after Rubin \cite{rubin_regular_1972}.

    The waiting times are exponentially distributed given the intensity,
i.e. their densities \(f_i(s)\) for any given transition are given
\[\begin{aligned}
\frac{d}{dt}P(t_{i+1}=t|\mathcal{F}(N(t)))
\end{aligned}\] fa \[\begin{aligned}
f_i(t)&\equiv f(t_{i+1}-t_i|\mathcal{F}(N(t_i)))\\
&=\Lambda(t_i,t|\mathcal{F}(N(t_i))\exp\left(-t\Lambda(t_i,t|\mathcal{F}(N(t_i))\right)
\end{aligned}\]

    It follows that the log likelihood \(\ell_i:=\log f_i\) of such a
waiting time \(s_i\) is given

\[\begin{aligned}
\ell_i(t_{i+1}-t_i) &=\log \Lambda(s_|\mathcal{F}(N(t_i)) - s_i \Lambda(s_i|\mathcal{F}(N(t_i))\\
&= \log \int_{t_i}^{t_{i+1}} \lambda(s|\mathcal{F}(N(t_i))) ds - {t_{i+1}} \int_{t_i}^{t_{i+1}} \lambda(s|\mathcal{F}(N(t_i))) ds
\end{aligned}
\]

    We may therefore write the likelihood for the data \[
\begin{aligned}
\ell(\mathbf{t}) &= \sum_{1\leq i\leq N(T)}\ell_i(t_{i+1}-t_i)\\
&= \sum_{1\leq i\leq N(T)} \left( \log \int_{t_i}^{t_{i+1}} \lambda(s|\mathcal{F}(N(t_i))) ds - {t_{i+1}} \int_{t_i}^{t_{i+1}} \lambda(s|\mathcal{F}(N(t_i))) ds\right)\\
&= \sum_{1\leq i\leq N(T)} \log \int_{t_i}^{t_{i+1}} \lambda(s|\mathcal{F}(N(t_i))) ds - \sum_{1\leq i\leq N(T)} {t_{i+1}} \int_{t_i}^{t_{i+1}} \lambda(s|\mathcal{F}(N(t_i))) ds\\
&= \sum_{1\leq i\leq N(T)} \log \int_{t_i}^{t_{i+1}} \lambda(s|\mathcal{F}(N(t_i))) ds - \sum_{1\leq i\leq N(T)} {t_{i+1}} \int_{t_i}^{t_{i+1}} \lambda(s|\mathcal{F}(N(t_i))) ds
\end{aligned}
\]

    The Hawkes process \cite{hawkes_point_1971} in particular, is the linear
self-exciting point process; Concretely, given a "background" rate
\(\mu:\mathbb{R}^+\rightarrow\mathbb{R}^+,\) and a non-negative
influence kernel with positive support
\(\phi:\mathbb{R}\rightarrow\mathbb{R}^+\cup\{0\}\), the Hawkes process
has the following expression for its intensity:

\[
\begin{aligned}
\lambda(t|\mathcal{F}(N(t);\mu,\phi) &= \mu(t) + \phi * N\\
&= \mu(t)  + \int_{-\infty}^{\infty}\phi(t-s)dN(s)\\
&= \mu(t) + \int_{-\infty}^{t}\phi(t-s)dN(s)\\
&= \mu(t) + \sum_{t_i\lt t}\phi(t-t_i)
\end{aligned}
\]

For brevity, we write
\(\lambda(\mathcal{F}(N(t));\mu,\phi)=\lambda^*(t)\), but will suppress
explicity dependence on history and parameters..

    The log-likelihood function for a given realisation of the data is can
be found from the usual likelihood for Poisson data of a given intensity
\cite{ozaki_etimating_1979}

\[\begin{aligned}L(N;\mu,\phi) &=-\int_0^T\lambda^*(t)dt + \int_0^T\log \lambda^*(t) dN_t \\
&=-\int_0^T\lambda^*(t)dt + \int_0^T\log \lambda^*(t) dN_t \\
&=-\int_0^T\lambda^*(t)dt + \sum_{t\in\mathbf{t}\cap [0,T]}\log \lambda^*(t)
\end{aligned}\]

Using the method of maximum likelihood we estimate our parameter vector
\(\theta:=(\mu, \phi)\) by choosing the estimates to maximise the
likelihood,
\[\hat{\theta}_\pi(N) = \operatorname{argmax}_{\mu,\phi} L_\pi(N;\theta).\]

    Here we extend the likelihood approach, treating it as penalised
functional regression problem. We augment the log-likeihood function
with a penalty functional \(\pi\):

\[L_\pi(N, \mu, \phi):= L(N; \mu, \phi)- \pi(\mu, \phi)\]

As in plain log-likelihood optimization, the estimate is defined as the
maximizer:

\[\hat{\theta}_\pi(N) = \operatorname{argmax}_\theta L_\pi(N;\theta)\]

    We take \(\mu\) to be an unknown sparse, non-random signal in a
particular sense. We will assume that, given a basis
\(\{\psi_j \}_{1\leq j\leq M}\), that we may write the background rate
as \[\mu(t) = \mu_0 + \sum_{j=1}^M  \omega_j \psi_j(t).\]
\emph{Sparsity} here will imply that the coefficients \(\omega_j\) will
be mostly 0.

In practice, we choose
\[\psi_j(t) := \mathbb{I}_{[\tau_{j},\tau_{j+1})}(t)\] for some lattice
\(\{\tau_j\},\) which will correspond to the assumption that "sparse"
triggers will have localised support in time; we are looking for short
"spikes".

Similarly, we will approximate \(\phi\) in a different functional basis
\(\{\bar{\psi}\}_{1\leq k \leq K}\), such that
\(\phi(t):= \sum_{k=1}^K \bar{\psi}(t;k)\). This basis choice is likely
arbitrary, but many exist in the literature trading off computational
and statistical convenience, including Laguerre polynomials
\cite{ogata_linear_1982}, exponential functions
\cite{schoenberg_consistent_2005,rambaldi_modeling_2015}, and step
functions \cite{eichler_graphical_2016} As a convenient choice, we
propose a basis of smooth unimodal non-negative functions with a
computationally convenient form, the Maxwell density,
\[\bar{\psi}_j(t;\sigma):=\sqrt{\frac{2}{\pi}} \frac{t^2\exp\left(-\frac{t^2}{2\sigma^2}\right)}{\sigma^3}.\]
so that

\[\phi(t):= \sum_{j=1}^K \kappa_j\bar{\psi}(t;\sigma_j).\] Once again,
some vector \(\{\sigma_j\}_{j\leq K}\) of time-scale parameters are
chosen \emph{a priori}.



    \begin{center}
    \adjustimage{max size={0.9\linewidth}{0.9\paperheight}}{chapter_sparse_hawkes_files/chapter_sparse_hawkes_15_0.pdf}
    \end{center}
    { \hspace*{\fill} \\}
    
    A motivation for these specific choices is a particular real-world
problem: modelling viral sharing and spreading in social media. We
suspect that online popularity is governed by some mixture of two
componenets:

\begin{enumerate}
\def\labelenumi{\arabic{enumi}.}
\tightlist
\item
  organic social sharing, modeled by the self-exciting \(\phi\)
  component, and
\item
  intermittent external boosts represented as variations in the the
  \(\mu\) component, such as when an internet meme is featured on an
  external medium such as television.
\end{enumerate}

Models such as the Hawkes process have frequently been applied in an
attempt to disambiguate these two
\cite{crane_robust_2008,mitchell_hawkes_2010,rizoiu_expecting_2017}.

    With basis functions fixed, the problem has become one of selection of
intercepts and basis expansions coefficients, and we may equivalently
write
\(\theta=(\mu, \kappa_1,\kappa_2,\dots,\kappa_K,\omega_1, \omega_2,\dots,\omega_M).\)

\(\ell_1\) type penalties are known for their ability to produce sparse
solutions, so we choose penalties of the form
\[\pi(\theta)=p\left\|\{\kappa_j\}\right\|_1+q\left\|\{\omega_j\}\right\|_1.\]
With the right choice of the parameters \(p\) and \(q\) we hope to
shrink unnecessary parameters to zero, identifying predictively
necessary ones, as in Lasso selection.

    There is no closed-form solution for the optimum of the Hawkes model
likelihood, penalised or not. However, we may use the gradient to find
it efficiently numerically. The optimisation proceeds using a standard
gradient-based black-box based Truncated Newton algorithm; as more
specialised sparse regression algorithms such as Least-Angle Regression,
do not directly apply here.

The gradient is efficient to calculate using standard standard
reverse-mode automatic differentiation \cite{baydin_automatic_2014} for
a wide variety of functional decomposition bases and penalties.

The enthusiast may also calculate the derivatives analytically. For
example, Ozaki \cite{ozaki_maximum_1979} give derivatives for the
parametric log likelihood estimator.

    Let \(\gamma\) be any of the components of the parameter vector
\(\theta\). Since the likelihood depends on this vector only through the
intensity process, we have

\[\begin{aligned}
{ \scriptstyle \frac{\partial}{\partial\gamma} } L_\pi(N;\theta)&=-\int_0^T{ \scriptstyle \frac{\partial}{\partial\gamma} }\lambda^*(t;\theta)dt + \int_0^T { \scriptstyle \frac{\partial}{\partial\gamma} }\log \lambda^*(t;\theta) dN_t -{ \scriptstyle \frac{\partial}{\partial\gamma} }\pi(\theta) \\
&=-\int_0^T{ \scriptstyle \frac{\partial}{\partial\gamma} }\lambda^*(t;\theta)dt + \sum_{0\leq t_i\leq T} \frac{{ \scriptstyle \frac{\partial}{\partial\gamma} }\lambda^*(t_i)}{\lambda^*(t_i)} -{ \scriptstyle \frac{\partial}{\partial\gamma} }\pi(\theta)
\end{aligned}
\]

    Taking \(\gamma=\mu\),

\[{ \scriptstyle \frac{\partial}{\partial\mu} }\lambda^*(t;\mu) = 1\]

I use higher derivatives for \(\mu\) so that I may optimize this
component using a higher order Newton method, since we know that its
optimum is typically shallow \cite{veen_estimation_2008}.

\[\begin{aligned}
  { \scriptstyle \frac{\partial}{\partial\mu} } L_\pi(N;\theta) &=-T + \sum_{0\leq t_i\leq T} \frac{1}{\lambda(t_i;\theta)} \\
  { \scriptstyle \frac{\partial^2}{\partial\mu^2} } L_\pi(N;\theta) &= \sum_{0\leq t_i\leq T} \frac{-1}{\lambda^2(t_i;\theta)}\\
  { \scriptstyle \frac{\partial^3}{\partial\mu^3} } L_\pi(N;\theta) &= \sum_{0\leq t_i\leq T} \frac{2}{\lambda^3(t_i;\theta)}
  \end{aligned}\]

    Now I handle \(\gamma=\kappa_j\),

\[{ \scriptstyle \frac{\partial}{\partial\kappa_j} }\lambda^*(t;\theta) =  \sum_{t_i\lt t}\phi_j(t-t_i)\]

so that

\[\begin{aligned}
  { \scriptstyle \frac{\partial}{\partial\kappa_i} } L_\pi(N; \theta)&=-\int_0^T \sum_{t_i\lt t}\phi_i(t-t_i) dt + \sum_{0\leq t_i\leq T} \frac{\sum_{t_k\lt t_i}\phi_j(t_i-t_k)}{\lambda_\theta(t_i)} -\frac{\partial}{\partial\kappa_j}\pi(\theta)
\end{aligned}\]

    Finally, I handle the \(\omega_j\) values.

Taking \(\theta=\omega_j\), and defining
\(\Delta\tau_j:= {\tau_{j-1}}-{\tau_j}\) we find

\[
\begin{align*}
{ \scriptstyle \frac{\partial}{\partial\omega_j} }\pi\|\boldsymbol \omega_\theta(t)\|_1 &= { \scriptstyle \frac{\partial}{\partial\omega_j} }\pi \int_{\tau_{j-1}}^{\tau_j} \left|\omega_j\right|dt\\
&=\pi\Delta\tau_j\operatorname{sgn} \omega_j\\
{ \scriptstyle \frac{\partial}{\partial\omega_j} }\lambda_\theta(t|\mathcal{F}_t) &=  \Delta\tau_j\mathbb{I}_{[\tau_{j-1},\tau_j)}(t)\\
{ \scriptstyle \frac{\partial}{\partial\omega_j} } L_\pi(\mathbf t;\theta) &=-\Delta\tau_j-\pi\Delta\tau_j\operatorname{sgn} \omega_j + \sum_{\tau_{j-1}\leq t_i\leq \tau_j} \frac{1}{\lambda_\theta(t_i)}
\end{align*}
\]

Higher partial derivatives are also analogous to the partial derivatives
with respect to \(\mu\), although of course the penalty introduces a
discontinuity at \(\omega_j=0\). This last formula is the key to the
gradient ascent algorithm.

Note that the \(\omega_j\) values are mutually orthogonal, in the sense
that \({ \scriptstyle \frac{\partial^2}{\partial\omega_i\omega_j}} =0\)
if \(i\ne j\). I can treat these components, more or less, as separate
univariate components when optimizing, and the Hessian will be sparse
off the diagonal.

By construction,
\({ \scriptstyle \frac{\partial}{\partial\mu} }\pi(\gamma) = { \scriptstyle \frac{\partial}{\partial\eta} }\pi(\gamma) = { \scriptstyle \frac{\partial}{\partial\kappa} }\pi(\gamma) = 0\).

TODO: I've changed \(\omega\) penalisation; this needs fixing.

    \section{Hyperparameter selection}\label{hyperparameter-selection}

    \subsubsection{Degrees of freedom}\label{degrees-of-freedom}

Whilst we might wish to select the regularisation penalties by cross
validation, cross validation is not straightforward for time series.

The traditional means of performing model selection in Hawkes-type
models e.g. \cite{ogata_estimation_1983}, uses \emph{degrees-of-freedom}
penalties \cite{efron_how_1886}. Specifically, Akaike's
\cite{akaike_likelihood_1981} Information Criterion, or \emph{AIC}, is
widely popular.

The AIC formula, for a given log-likelihood parameter estimate
\(\hat{\theta}(N)\), with 'degrees of freedom', \(d\) is given

\[\mathrm{AIC}(N,\hat{\theta}(N))=2d-2L(N,\hat{\theta}(N))\]

The degrees of freedom for a parametric model is taken to be the number
of free parameters, equal to the cardinality of the parameter vector
\(\theta\).

    The AIC does not immediately solve the problem of regularized parameter
selection where the regularisation parameter may be a real value
smoothly interpolating many possible models. One can calculate an AIC
for penalised models, but this require onerous Hessian matrix inversions
and becomes costly in high dimensions, and in any case does not
generalise readily to non-smooth penalties.
\cite{konishi_information_2008}

We proceed heuristically, based on the degrees-of-freedom arguments from
Lasso-type linear-model regularisation. There, the number of non-zero
coefficients fit under a particular value of the regularization penalty
\(\pi\) in an \(\ell_1\) regression model provides an unbiased estimate
of the effective degrees of freedom of the model
\cite{zou_degrees_2007}.

    This suggests, by analogy, an AIC estimate based on the number of
non-zero parameters for given regularization coefficients:

\[
\hat{d}\left(N,\hat{\theta},p,q\right)=\sum_j\mathbb 1\left\{\hat{\theta_j}(N,p,q)\neq 0\right\}
\]

    The suggested combined procedure is to combine, in an inner loop, the
fitting of model parameters for given values of \(p\) and \(q\), and
selecting the hyperparameters with the best overall AIC value.

    \subsection{Simulations}\label{simulations}

    TODO: switch to consistent step-function display of instantaneous rate.

    We simulate data with parameters to match the stylized behaviour of the
social media time series of previous work
\cite{crane_robust_2008,mitchell_hawkes_2010,rizoiu_expecting_2017}, to
test the efficacy of the proposed method.



    \begin{center}
    \adjustimage{max size={0.9\linewidth}{0.9\paperheight}}{chapter_sparse_hawkes_files/chapter_sparse_hawkes_31_0.pdf}
    \end{center}
    { \hspace*{\fill} \\}
    
    \begin{center}
    \adjustimage{max size={0.9\linewidth}{0.9\paperheight}}{chapter_sparse_hawkes_files/chapter_sparse_hawkes_31_1.pdf}
    \end{center}
    { \hspace*{\fill} \\}
    
    \begin{center}
    \adjustimage{max size={0.9\linewidth}{0.9\paperheight}}{chapter_sparse_hawkes_files/chapter_sparse_hawkes_31_2.pdf}
    \end{center}
    { \hspace*{\fill} \\}
    
    Extensive grid search for optimsal parameters is expensive, so we select
both regularisation parameters using optimal sequential sampling by
Gaussian process regression of the combined objective function
\cite{snoek_practical_2012} using the software package
\href{https://github.com/scikit-optimize/scikit-optimize}{scikit-optimize}.


\texttt{\color{outcolor}Out[{\color{outcolor}27}]:}
    
    \begin{center}
    \adjustimage{max size={0.9\linewidth}{0.9\paperheight}}{chapter_sparse_hawkes_files/chapter_sparse_hawkes_33_0.png}
    \end{center}
    { \hspace*{\fill} \\}
    

    With a single regularisation parameter, only for the background rate
coefficient \(\omega_j\), we recover the background rate efficiently.


\texttt{\color{outcolor}Out[{\color{outcolor}28}]:}
    
    \begin{center}
    \adjustimage{max size={0.9\linewidth}{0.9\paperheight}}{chapter_sparse_hawkes_files/chapter_sparse_hawkes_35_0.png}
    \end{center}
    { \hspace*{\fill} \\}
    

    \subsection{How to simulate a Hawkes
process}\label{how-to-simulate-a-hawkes-process}

Here we follow a standard "thinning" method of simulating Hawkes
processes, specialised from Algorithm 7.5.IV of Daley and Vere-Jones'
\emph{Introduction to the Theory of Point Processes Volume 1}
\cite{daley_introduction_2003}, restating Ogata's original Algorithm 2
(\cite{ogata_lewis_1981}.

Since that is a general case, here we present, informally, a simpler,
specific version for linear Hawkes processes.

For clarity, we will use the representation of the Hawkes process as an
immigrant-birth process, where the recursive "self-excited" component
corresponds to the "births". We will assume that we are given a list of
arrival times of immigrants, \(\{a_1,a_2,\dots,a_N\}\), and wish to
simulate the births arising from these immigrants. We will associate a
point and its arrival time and discuss them interchangeably where
ambiguity does not arise.

Each of these \(a_i\) gives rise to a family of descendants
\(\mathrm{descendants}(a_i)\), distributed independently of all other
families, whose combined superposition is the final Hawkes process
(\cite{hawkes_cluster_1974}. It will suffice to simulate the arrival
times of each of these families. Without loss of generality we may
consider the problem of simulating the births in a single family started
at time \(a_1=0\) and translate it.

Now the families
\(\mathrm{descendants}(t_j)\forall t_j\in \mathrm{children}(a_i)\) are
all once again identically and mutually independently distributed.
Proceeding by induction, it suffices to simulate the children of a
single point at time \(0\). The same method will suffice for its
children, children of children and so on. The final realisation can be
constructed from these simulations by appropriate concatenation and
translation of the generations of each family.

The single kernel simulation is simple, since the Poisson intensity
process is no longer conditional on the history of the sequence. As
before, the influence kernel is \(\phi\). We will assume that \(\phi\)
is a.e. continuous \(L_1\) integrable and eventually decreasing in
\(t\). We associate with \(\phi\) a "majorant", \(\phi^{**}\) such that
\[\phi^{**}(t)\geq\phi(t), \forall t>0\] and \(\phi^{**}\) is
non-increasing and integrable. As an aside, we mention that for the
unimodal influence kernels used in the current paper, such majorants are
trivial to construct.

We will use the fact that there is some time \(T>0\) such that we do not
consider events after it, i.e. that we only care about arrival times on
\([0,T]\), and so only children that occur before time \(T\) can be
relevent.

\subsubsection{Algorithm: Ogata's thinning
algorithm}\label{algorithm-ogatas-thinning-algorithm}

Our goal is to simulate the set \(S=\mathrm{children}(a_0)\).

\begin{enumerate}
\def\labelenumi{\arabic{enumi}.}
\tightlist
\item
  Set \(i=1, t=0\), \(S=\varnothing\)
\item
  Generate \(\Delta\sim \operatorname{Exp}(1/\phi^{**}(t))\)
\item
  Set \(t\leftarrow t+\Delta\)
\item
  If \(t\geq T\), halt.
\item
  With probability \(P=\frac{\phi(t)}{\phi^{**}(t)}\), set
  \(S\leftarrow S\cup \{t\}\).
\item
  Continue from step 2.
\end{enumerate}

    \subsection{Conclusions}\label{conclusions}

    

    \subsection{Appendix and nonsense}\label{appendix-and-nonsense}


    Given two ML fitted models, \((M_1,\hat{\theta}^M_1)\) and
\((M_2\,\hat{\theta}^M_2)\), the difference in their AIC values is an
estimate of the relative Kullback-Leibler divergence of the inferred
measures \(\hat{\mu_1},\hat{\mu_2}\) from the unknown true distribution,
\(\mu\) i.e.

\[\mathrm{AIC}(X,M_1)-\mathrm{AIC}(X,M_2) \simeq D_\mathrm{KL}(\mu\|\hat{\mu_1}) - D_\mathrm{KL}(\mu\|\hat{\mu_2})\]

    mean delay \(\mu=2\tau \sqrt{\frac{2}{\pi}}\)

mode \(\sqrt{2}\tau\)

scale \(\sigma=\sqrt{\frac{\tau^2(3 \pi - 8)}{\pi}}\)

We take fixed \(M\)-point lattice \(\tilde{\tau}\) and \(K\)-fixed
scales \(\tilde{\sigma}.\)

The parameter vector to be estimated then is
\(\theta:=(\mu, \tilde{\omega}, \tilde{\kappa}).\)

    The AIC is based on asymptotic arguments, and Sugiura's
finite-sample-corrected version \emph{AICc}, has lower bias for most
practical applications \cite{sugiura_further_1978}.

\[{\mathrm {AICc}}:={\mathrm {AIC}}+{\frac {2d(d+1)}{N_T-d-1}}\]

    For non-parametric extension to a parametric model, one usually
penalizes only the non-parametric extensions to the model, such that
they vanish as the penalty term increases. \cite{green_penalized_1987}

\[\lim_{\pi\to\infty}\hat{\theta}_\pi'(\mathbf t) =\hat{\theta}(\mathbf t)\]

I follow this practice here, penalizing only the deviations of
\(\omega(t)\) from the parametric estimate.

\[L_\pi(\mathbf t;\mu,\phi):=-\int_0^T\lambda(t;\mu,\phi)dt + \int_0^T\log \lambda(t;\mu,\phi) dN_t - \pi\|\boldsymbol \omega_\theta(t)\|_1\]


    % Add a bibliography block to the postdoc
    
    

\bibliographystyle{apacite}
\bibliography{refs}

    
    \end{document}
